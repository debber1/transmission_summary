% !TeX spellcheck = en_GB
\documentclass[../transmission.tex]{subfiles}
\begin{document}
	\section{The lossless transmission line}
		\subsection{Parameters of the lossless line}
			For a \textbf{lossless transmission line}, we say $R' = G' = 0$ in figure \ref{fig:trans_line_lumped_model}. This leads to the following simplifications:
			\begin{itemize}
				\item The \textbf{complex propagation constant} becomes $\gamma = j\omega\sqrt{L'C'}$, meaning that $\alpha$ (the attenuation constant) is 0 and $\beta$ (the phase constant) is $\omega\sqrt{L'C'}$.
				\item The \textbf{characteristic impedance} becomes $Z_0 = \sqrt{\frac{L'}{C'}}=R_0$, a real-valued parameter.
			\end{itemize}
			Here, $L'$ is dependent on $\mu_r$, which is the relative permeability of the dielectric between the two transmission line conductors. $C'$ is dependent on $\epsilon_r$ which is the relative permittivity of the dielectric between the conductors.
			\subsubsection{Characteristic impedance $R_0$}
				The characteristic impedance depends on the geometry of the conductors. We can find three relations which will (without proof) hold true for a lossless transmission line:
				\begin{itemize}
					\item $\sqrt{\frac{\mu_0}{\epsilon_0}}=120\pi$
					\item $R_0\sim \frac{1}{\sqrt{\epsilon_r}}$
					\item $\mu_0\epsilon_0=\frac{1}{c^2}$
				\end{itemize}
				Using these relations, and the assumption that $\mu_r\approx1$ we get:
				\begin{equation}
					\beta = \omega\sqrt{L'C'}=2\pi\frac{f}{c}\sqrt{\epsilon_r}=2\pi\frac{\sqrt{\epsilon_r}}{\lambda_0}
				\end{equation}
				for the phase constant $\beta$.
		
		\subsection{Voltage and current on the lossless line}
			\label{sec:volt+curr_lossless_line}
			When we plug these simplifications into equations derived in section \ref{sec:phasor_representation}, we get the following:
			\begin{align}
				v(x,t)&=Re[Ve^{j\omega t}]\\
				&=\underbrace{V^+_m\cos(\beta x+\omega t +\varphi_+)}_{\texttt{1st term}}+\underbrace{V_m^-\cos(-\beta x+\omega t+\varphi_-)}_{\texttt{2nd term}}\\
				i(x,t)&=Re[Ie^{j\omega t}]\\
				&=\frac{V^+_m}{R_0}\cos(\beta x+\omega t +\varphi_+)-\frac{V_m^-}{R_0}\cos(-\beta x+\omega t+\varphi_-)
			\end{align}
			This means that $v(x,t)$ and $i(x,t)$ consist of two superimposed waves travelling with constant amplitude and opposite propagation direction along the line.
			
			\subsubsection{The direct voltage wave}
				The first term in the voltage and current equations represents \textbf{the direct voltage wave}, some of its properties are:
				\begin{enumerate}
					\item The amplitude $V^+_m$ is constant: \textbf{no attenuation}.
					\item The wave is \textbf{sinusoidal}, the wavelength is the smallest distance $x$ between two points of the same amplitude when $t$ is a constant. We also find that $\beta\lambda = 2\pi$. This leads to the formula $\lambda =\lambda_0/\sqrt{\epsilon_r}$. Here, we define $\lambda_0 = \frac{c}{f}$. We see that $\lambda_0$ is always smaller than $\lambda$ by a factor of $1/\sqrt{\epsilon_r}$, this factor is called \textbf{the velocity factor (VF)}.
					\item We can also define the \textbf{phase velocity} $v_p$.  where the relation $v_p=-c/\sqrt{\epsilon_r}$ holds true. We can again see the velocity factor popping up here.
				\end{enumerate}
				For this wave, $\beta$ indicates the phase change per unit length and $\omega$ indicates the phase change per unit time. The direct wave is represented in figure \ref{fig:chap02losslessdirectwave}.
				
				\begin{figure}[h]
					\centering
					\includegraphics[width=\linewidth]{../assets/chap02_lossless_direct_wave.png}
					\caption[The direct voltage wave]{The direct voltage wave}
					\label{fig:chap02losslessdirectwave}
				\end{figure}
			
			\subsubsection{The direct current wave}
				The first term of the second equation in \ref{sec:volt+curr_lossless_line}, is called the \textbf{direct current wave}. This wave has the same properties as the direct voltage wave. 
			
			\subsubsection{The indirect waves}
				The second term of both equations are called the \textbf{indirect or reflected voltage and current wave}. They also have similar properties to the direct voltage wave, but they travel in the opposite direction. They are represented by the second term of each equation if section \ref{sec:volt+curr_lossless_line}.
				
			\subsubsection{The standing wave}
				The standing wave in a transmission line is the result of the superposition of a direct and indirect travelling wave with opposite directions of propagation.  Figure \ref{fig:chap02_standing_wave} is an example of these standing waves, each line depicts a voltage along the line at a different time instance.  The point at which the envelope of the lines reaches a maximum is called an \textbf{anti-node} and the minimum is called a \textbf{node}.
				
				\begin{figure}[h]
					\centering
					\includegraphics[width=\linewidth]{../assets/chap02_standing_wave.png}
					\caption{Example of a standing wave pattern on a transmission line}
					\label{fig:chap02_standing_wave}
				\end{figure}
		
		\subsection{The voltage reflection coefficient $\Gamma$}
			The \textbf{reflection coefficient} $\Gamma$ is defined as the ratio of the phasor of the reflected voltage wave to the phasor of the direct voltage wave. It can be defined at any point on the line and is thus a function of $x$.
			
			The modulus of $\Gamma$ expresses the \textbf{amplitude ratio} of the direct and indirect wave, while the phase of $\Gamma$ indicates the \textbf{phase difference} between the two. At places where the phase difference is an even multiple of $\pi$ we get constructive interference, and thus an anti-node in figure \ref{fig:chap02_standing_wave}, an odd multiple gives us destructive interference and thus we get a node. \\
			\\
			Finally, we have the following formulas for calculating $\Gamma$:
			\begin{equation}
				\Gamma(x=0) = \frac{Z_L-R_0}{Z_L+R_0} = \Gamma_L
			\end{equation}
			and
			\begin{equation}
				\Gamma(x)=\Gamma_Le^{-j2\beta x}
			\end{equation}
			
			\subsubsection{The reflection coefficient in the complex plain}
				Figure \ref{fig:chap02_gamma_complex} represents $\Gamma$ in the complex plane. As expected from a lossless line, the amplitude ratio (modulus of $\Gamma$) \textbf{does not change} under any circumstance. However, when we increase $x$, we see that the phase difference increases, this manifests itself as \textbf{drawing a circle in the complex plane}. Finally, when $x$ increases by half a wave length, $\Gamma$ describes a full circle:
				\begin{equation}
					2\beta\Delta x= 2\pi \quad\Rightarrow\quad 2\frac{2\pi}{\lambda}\Delta x = 2\pi \quad\Rightarrow\quad \Delta x = \frac{\lambda}{2}
				\end{equation}
				\begin{figure}[h]
					\centering
					\includegraphics[width=0.5\linewidth]{../assets/chap02_gamma_complex.png}
					\caption{$\Gamma$ in the complex plane as a function of $x$}
					\label{fig:chap02_gamma_complex}
				\end{figure}
				
				An important side-note is that, for \textbf{passive loads} the modulus of $\Gamma$ or $\Gamma_L$ is \textbf{always less than or equal to 1}. This means that the amplitude $V_m^-$ of the reflected wave, is less than or equal to the one of the direct wave $V_m^+$.
				
			\subsubsection{Expressing voltage and current with the reflection coefficient}
				The phasor of the voltage and current can be noted with the help of $\Gamma$ as:
				\begin{align}
					V&=V^+e^{j\beta x}[1+\Gamma]\\
					I&=\frac{V^+}{R_0}e^{j\beta x}[1-\Gamma]
				\end{align}
			
			\subsubsection{The impedance of a piece of transmission line}
				We can calculate the impedance $Z_x$ which is seen at the location $x$ on the line in the direction of the load as follows:
				\begin{align}
					Z_x &= \frac{V}{I} = R_0\frac{1+\Gamma}{1-\Gamma}\\
					\Gamma&=\frac{Z_x-R_0}{Z_x+R_0}
				\end{align}
				
				\paragraph{The matched line}
					We get a \textbf{matched line} when $Z_L = R_0$, in this state the reflection coefficient becomes zero $\Gamma = \Gamma_L=0$. This means that there is \textbf{no reflected wave}. Often, this is a desirable situation in practice. Furthermore, this means that $Z_x = R_0$ independent of $x$.\\
					
					We can also derive a possible definition of $R_0$ as a result of this phenomenon: ``the characteristic impedance of a lossless line is the impedance that one measures in each point of the line when the line is closed with this impedance $R_0$''.
				
				\paragraph{A note on impedances}
					$Z_x$ is the ratio  of the \textbf{total} voltage (sum of direct and indirect waves) to the \textbf{total} current. This contrasts with the characteristic impedance of the line which gives the relationship betwen voltage and current of each wave  separately $R_0 = \frac{V^+}{I^+} = -\frac{V^-}{I^-}$. $R_0$ is the impedance ``seen'' by the travelling waves,  $Z_x$ is the impedance ``seen'' by the standing wave. 
				
\end{document}
