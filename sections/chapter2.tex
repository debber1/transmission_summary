% !TeX spellcheck = en_GB
\documentclass[../transmission.tex]{subfiles}
\begin{document}
	\section{The lossless transmission line}
		\subsection{Parameters of the lossless line}
			For a \textbf{lossless transmission line}, we say $R' = G' = 0$ in figure \ref{fig:trans_line_lumped_model}. This leads to the following simplifications:
			\begin{itemize}
				\item The \textbf{complex propagation constant} becomes $\gamma = j\omega\sqrt{L'C'}$, meaning that $\alpha$ (the attenuation constant) is 0 and $\beta$ (the phase constant) is $\omega\sqrt{L'C'}$.
				\item The \textbf{characteristic impedance} becomes $Z_0 = \sqrt{\frac{L'}{C'}}=R_0$, a real-valued parameter.
			\end{itemize}
			Here, $L'$ is dependent on $\mu_r$, which is the relative permeability of the dielectric between the two transmission line conductors. $C'$ is dependent on $\epsilon_r$ which is the relative permittivity of the dielectric between the conductors.
			\subsubsection{Characteristic impedance $R_0$}
				The characteristic impedance depends on the geometry of the conductors. We can find three relations which will (without proof) hold true for a lossless transmission line:
				\begin{itemize}
					\item $\sqrt{\frac{\mu_0}{\epsilon_0}}=120\pi$
					\item $R_0\sim \frac{1}{\sqrt{\epsilon_r}}$
					\item $\mu_0\epsilon_0=\frac{1}{c^2}$
				\end{itemize}
				Using these relations, and the assumption that $\mu_r\approx1$ we get:
				\begin{equation}
					\beta = \omega\sqrt{L'C'}=2\pi\frac{f}{c}\sqrt{\epsilon_r}=2\pi\frac{\sqrt{\epsilon_r}}{\lambda_0}
				\end{equation}
				for the phase constant $\beta$.
		
		\subsection{Voltage and current on the lossless line}
			When we plug these simplifications into equations derived in section \ref{sec:phasor_representation}, we get the following:
			\begin{align}
				v(x,t)&=Re[Ve^{j\omega t}]\\
				&=\underbrace{V^+_m\cos(\beta x+\omega t +\varphi_+)}_{\texttt{1st term}}+\underbrace{V_m^-\cos(-\beta x+\omega t+\varphi_-)}_{\texttt{2nd term}}\\
				i(x,t)&=Re[Ie^{j\omega t}]\\
				&=\frac{V^+_m}{R_0}\cos(\beta x+\omega t +\varphi_+)-\frac{V_m^-}{R_0}\cos(-\beta x+\omega t+\varphi_-)
			\end{align}
			This means that $v(x,t)$ and $i(x,t)$ consist of two superimposed waves travelling with constant amplitude and opposite propagation direction along the line.
		
\end{document}
