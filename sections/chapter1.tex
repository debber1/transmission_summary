% !TeX spellcheck = en_GB
\documentclass[../transmission.tex]{subfiles}
\begin{document}
	\section{Introduction to transmission lines}
		\subsection{What is a transmission line?}
			In the most egeral case, a transmission line is a structure used to transport energy or information between two points. The most general representation of it is shown in figure \ref{fig:trans_line_2port}, where a \textbf{two-port-network} is represented. The left-most port contains a generator circuit and is known as the \textbf{transmitting side}. On the far right the \textbf{receiving side} is represented by a load circuit. As an aside, the generator circuit can always be represented by a Thévenin equivalent.
			
			\begin{figure}[h]
				\begin{center}
				\begin{circuitikz}[american]
					%Generator circuit
					\draw (0, 0) to[vsource, v=$V_g$,invert] (0,2) to[generic,l=$Z_g$] (2,2) node[ocirc, label=above:1]{}--(3,2)(3,0)--(2,0)node[ocirc, label=below:1']{}--(0,0);
					\draw[dashed] (-1,-1) rectangle (1.8,3) (0.5,-1.3)node[]{Generator};
					
					%Transmission line box
					\draw(3,2.2)--(7,2.2)--(7,-0.2)--(3,-0.2)--(3,2.2) (5,1)node[]{Transmission line};
					
					%Load circuit
					\draw(7,2)--(8,2)node[ocirc, label=above:2]{}--(10,2)to[generic,l=$Z_L$](10,0)--(8,0)(7,0)--(8,0)node[ocirc, label=below:2']{};
					\draw[dashed] (8.2,-1) rectangle (11,3) (9.5,-1.3)node[]{Load};
				\end{circuitikz}
				\end{center}
				\caption{Transmission line two-port representation}
				\label{fig:trans_line_2port}
			\end{figure}
			The transmitted power can range from a few µW (Ethernet) to several kW (AM radio). The characteristics of the transmission line depends on \textbf{two factors}: the frequency f of the signal and the length L of the line. 
\end{document}
