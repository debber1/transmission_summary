% !TeX spellcheck = en_GB
\documentclass[../transmission.tex]{subfiles}
\begin{document}
	\section{Introduction to transmission lines}
		\subsection{What is a transmission line?}
			In the most egeral case, a transmission line is a structure used to transport energy or information between two points. The most general representation of it is shown in figure \ref{fig:trans_line_2port}, where a \textbf{two-port-network} is represented. The left-most port contains a generator circuit and is known as the \textbf{transmitting side}. On the far right the \textbf{receiving side} is represented by a load circuit. As an aside, the generator circuit can always be represented by a Thévenin equivalent.
			
			\begin{figure}[h]
				\begin{center}
				\begin{circuitikz}[american]
					%Generator circuit
					\draw (0, 0) to[vsource, v=$V_g$,invert] (0,2) to[generic,l=$Z_g$] (2,2) node[ocirc, label=above:1]{}--(3,2)(3,0)--(2,0)node[ocirc, label=below:1']{}--(0,0);
					\draw[dashed] (-1,-1) rectangle (1.8,3) (0.5,-1.3)node[]{Generator};
					
					%Transmission line box
					\draw(3,2.2)--(7,2.2)--(7,-0.2)--(3,-0.2)--(3,2.2) (5,1)node[]{Transmission line};
					
					%Load circuit
					\draw(7,2)--(8,2)node[ocirc, label=above:2]{}--(10,2)to[generic,l=$Z_L$](10,0)--(8,0)(7,0)--(8,0)node[ocirc, label=below:2']{};
					\draw[dashed] (8.2,-1) rectangle (11,3) (9.5,-1.3)node[]{Load};
				\end{circuitikz}
				\end{center}
				\caption{Transmission line two-port representation}
				\label{fig:trans_line_2port}
			\end{figure}
			The transmitted power can range from a few µW (Ethernet) to several kW (AM radio). The characteristics of the transmission line depends on \textbf{two factors}: the frequency f of the signal and the length L of the line. 
			
		\subsection{Why do we use transmission lines?}
			We need transmission lines because signals travel at finite speeds (usually $c=3\cdot10^8$). This introduces a delay in terminal 22' compared to 11' in figure \ref{fig:trans_line_2port}, essentially introducing a time shift of $L/c$ as seen in figure \ref{fig:delay_trans_line}.
			
			\begin{figure}[h]
				\begin{center}
					\begin{tikzpicture}[xscale=3,yscale=2]
						\draw[->] (0,-1.3) -- (0,1.3);
						\draw[->] (0,0)--(4.3,0);
						\draw[domain=-0.1:4.3,samples=100,black] plot(\x,{sin(2*\x r)});
						\draw[domain=-0.1:4.3,samples=100,black, dashed] plot(\x,{sin((2*\x-1) r)});
						\draw[dashed] (0,0) -- (0,1.3);
						\draw[dashed] (1/2,0) -- (1/2,1.3);
						\draw[<->] (0,1.15) -- node[above]{\small $delay$} (1/2,1.15);
						\draw(0.9,1.2)node[]{$V_ {11'}(t)$};
						\draw(1.6,1.2)node[]{$V_ {22'}(t)$};
					\end{tikzpicture}	
				\end{center}
				\caption{Delay between $V_{11'}(t)$ and $V_{22'}(t)$}
				\label{fig:delay_trans_line}
			\end{figure}
			Mathematically this comes down to:
			\begin{align}
				V_{22'}(t) &= V_{11'}(t-\underbrace{L/c}_{delay})\\
				&= V_0\cos[\omega(t-L/c)]\\
				&=V_0\cos[\omega t-\underbrace{\omega L/c}_{phase shift}]
			\end{align}
			This gives us the \textbf{phase shift} $\omega L/c$, which is the determining factor for whether you need a  transmission line or not:
			\begin{equation}
				\frac{\omega L}{c} = \frac{2\pi fL}{c} = 2\pi\frac{L}{\lambda}
			\end{equation}
			In the end, it is \textbf{the ratio} of the physical line dimensions and the wave length $L/\lambda$ which determine the impact of the transmission line.  When it is very small, the effects can be neglected. However, when $L/\lambda > 0.01$ it becomes necessary to take into account not only the phase shift but also the other effects which will be handled in further chapters:
			\begin{enumerate}
				\item reflected signals
				\item power loss
				\item dispersion
			\end{enumerate} 
\end{document}
