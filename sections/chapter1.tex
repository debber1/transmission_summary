% !TeX spellcheck = en_GB
\documentclass[../transmission.tex]{subfiles}
\begin{document}
	\section{Introduction to transmission lines}
		\subsection{What is a transmission line?}
			In the most egeral case, a transmission line is a structure used to transport energy or information between two points. The most general representation of it is shown in figure \ref{fig:trans_line_2port}, where a \textbf{two-port-network} is represented. The left-most port contains a generator circuit and is known as the \textbf{transmitting side}. On the far right the \textbf{receiving side} is represented by a load circuit. As an aside, the generator circuit can always be represented by a Thévenin equivalent.
			
			\begin{figure}[h]
				\begin{center}
					\begin{circuitikz}[american]
						%Generator circuit
						\draw (0, 0) to[vsource, v=$V_g$,invert] (0,2) to[generic,l=$Z_g$] (2,2) node[ocirc, label=above:1]{}--(3,2)(3,0)--(2,0)node[ocirc, label=below:1']{}--(0,0);
						\draw[dashed] (-1,-1) rectangle (1.8,3) (0.5,-1.3)node[]{Generator};
						
						%Transmission line box
						\draw(3,2.2)--(7,2.2)--(7,-0.2)--(3,-0.2)--(3,2.2) (5,1)node[]{Transmission line};
						
						%Load circuit
						\draw(7,2)--(8,2)node[ocirc, label=above:2]{}--(10,2)to[generic,l=$Z_L$](10,0)--(8,0)(7,0)--(8,0)node[ocirc, label=below:2']{};
						\draw[dashed] (8.2,-1) rectangle (11,3) (9.5,-1.3)node[]{Load};
					\end{circuitikz}
				\end{center}
				\caption{Transmission line two-port representation}
				\label{fig:trans_line_2port}
			\end{figure}
			The transmitted power can range from a few µW (Ethernet) to several kW (AM radio). The characteristics of the transmission line depends on \textbf{two factors}: the frequency f of the signal and the length L of the line. 
			
		\subsection{Why do we use transmission lines?}
			We need transmission lines because signals travel at finite speeds (usually $c=3\cdot10^8$). This introduces a delay in terminal 22' compared to 11' in figure \ref{fig:trans_line_2port}, essentially introducing a time shift of $L/c$ as seen in figure \ref{fig:delay_trans_line}.
			
			\begin{figure}[h]
				\begin{center}
					\begin{tikzpicture}[xscale=3,yscale=2]
						\draw[->] (0,-1.3) -- (0,1.3);
						\draw[->] (0,0)--(4.3,0);
						\draw[domain=-0.1:4.3,samples=100,black] plot(\x,{sin(2*\x r)});
						\draw[domain=-0.1:4.3,samples=100,black, dashed] plot(\x,{sin((2*\x-1) r)});
						\draw[dashed] (0,0) -- (0,1.3);
						\draw[dashed] (1/2,0) -- (1/2,1.3);
						\draw[<->] (0,1.15) -- node[above]{\small $delay$} (1/2,1.15);
						\draw(0.9,1.2)node[]{$V_ {11'}(t)$};
						\draw(1.6,1.2)node[]{$V_ {22'}(t)$};
					\end{tikzpicture}	
				\end{center}
				\caption{Delay between $V_{11'}(t)$ and $V_{22'}(t)$}
				\label{fig:delay_trans_line}
			\end{figure}
			Mathematically this comes down to:
			\begin{align}
				V_{22'}(t) &= V_{11'}(t-\underbrace{L/c}_{delay})\\
				&= V_0\cos[\omega(t-L/c)]\\
				&=V_0\cos[\omega t-\underbrace{\omega L/c}_{phase shift}]
			\end{align}
			This gives us the \textbf{phase shift} $\omega L/c$, which is the determining factor for whether you need a  transmission line or not:
			\begin{equation}
				\frac{\omega L}{c} = \frac{2\pi fL}{c} = 2\pi\frac{L}{\lambda}
			\end{equation}
			In the end, it is \textbf{the ratio} of the physical line dimensions and the wave length $L/\lambda$ which determine the impact of the transmission line.  When it is very small, the effects can be neglected. However, when $L/\lambda > 0.01$ it becomes necessary to take into account not only the phase shift but also the other effects which will be handled in further chapters:
			\begin{enumerate}
				\item reflected signals
				\item power loss
				\item dispersion
			\end{enumerate} 
			
		\subsection{Propagation modes}
			There are two common types of transmission lines:
			\begin{description}
				\item[Transverse electro-magnetic (TEM) lines] Waves propagating along these lines are characterized by electric and magnetic fields that are completely transverse (perpendicular) to the direction of propagation. Examples are coax cables and traces on PCB's. 
				\item[Higher-order transmission lines] Waves propagating along these types of lines have at least one significant field component in the direction of propagation. Wave guides and fibre optic cables are examples of this mode.
			\end{description}
			This course only deals with TEM modes.
		
		\newpage	
		\subsection{Generalized line equations}
			\subsubsection{Transmission line parameters}
				In figure \ref{fig:trans_line_parameters}, we can see the model of a transmission line which we use to derive the generalized line equations. Notice that the voltage and current in the line is not only time based, but also location based. A further quirk of this model is that the x-axis is reversed to the normal convention. 
				
				\begin{figure}[h]
					\begin{center}
						\begin{circuitikz}[american]
							%Generator circuit
							\draw (0, 0) to[vsource, v=$V_g$,invert] (0,2) to[generic,l=$Z_g$] (2,2) node[ocirc, label=above:1]{}--(3,2)(3,0)--(2,0)node[ocirc, label=below:1']{}--(0,0);
							\draw[dashed] (-1,-1) rectangle (1.8,3) (0.5,-1.3)node[]{Generator};
							
							%Transmission line box
							\draw(3,2)-- (4,2) to[short, -, i=$I_{x,t}$] (6,2)--(7,2) (3,0)--(7,0);
							\draw[<-] (1,-1.7) -- (9,-1.7) ;
							\draw (2,-1.6)--(2,-1.8) (8,-1.6)--(8,-1.8);
							\draw(2,-2.1)node[]{$x=L$};
							\draw(8,-2.1)node[]{$x=0$};
							\draw(5.3,1)node[]{$V_{x,t}$};
							\draw[<-] (4.9,2) -- (4.9,0) ;
							%Load circuit
							\draw(7,2)--(8,2)node[ocirc, label=above:2]{}--(10,2)to[generic,l=$Z_L$](10,0)--(8,0)(7,0)--(8,0)node[ocirc, label=below:2']{};
							\draw[dashed] (8.2,-1) rectangle (11,3) (9.5,-1.3)node[]{Load};
						\end{circuitikz}
					\end{center}
					\caption{General depiction of a transmission line}
					\label{fig:trans_line_parameters}
				\end{figure}
				We can split up the x-axis into infinitesimally small \textbf{lumped-element} models as seen in figure \ref{fig:trans_line_lumped_model}. This gives rise to four basic elements which we call \textbf{transmission line parameters}:
				\begin{itemize}
					\item $R'$: Resistance per meter of the two conductors together $[\Omega/m]$
					\item $L'$: Inductance per meter of the two conductors together $[H/m]$
					\item $C'$: Capacity per meter of the two conductors together $[F/m]$
					\item $G'$: Conductance per meter between the two conductors $[S/m]$
				\end{itemize}
				\begin{figure}[h]
					\begin{center}
						\begin{circuitikz}[american]
							\draw(0,2)node[ocirc]{}to[R, l=$R'dx$](2,2)to[L,l=$L'dx$](4,2)to[R,l_=$G'dx$](4,0)--(0,0)node[ocirc]{} (4,2)--(5,2)to[C,l=$C'dx$](5,0)--(4,0);
							\draw (5,2)--(6,2)node[ocirc]{} (5,0)--(6,0)node[ocirc]{} ;
							\draw(6,2)node[ocirc]{}to[R, l=$R'dx$](8,2)to[L,l=$L'dx$](11,2)to[R,l_=$G'dx$](11,0)--(6,0)node[ocirc]{} (11,2)--(12,2)to[C,l=$C'dx$](12,0)--(10,0);
							\draw(3,-0.3)node[]{$dx$};
							\draw(9,-0.3)node[]{$dx$};
							\draw (12,2)--(13,2);
							\draw (12,0)--(13,0);
						\end{circuitikz}
					\end{center}
					\caption{Equivalent circuit for a transmission line}
					\label{fig:trans_line_lumped_model}
				\end{figure}
				
			\subsubsection{Skin effect}
				The resistance $R'$ is always higher than the DC resistance of the conductor. This is due to the \textbf{skin effect} which occurs in AC currents. Essentially, it means that the current will not flow through the entire wire, but only a section of it. If the frequency increases, the section shrinks. For a circular conductor, we can define the following equation:
				\begin{equation}
					J = J_m e^{-x/\delta}
				\end{equation}
				Where $J_m$ is the density at the surface of the conductor and x is the radial distance from the surface. The \textbf{skin depth} $\delta$ can be calculated as $\delta = \sqrt{\frac{2\rho}{2\pi f\mu}}$ in most cases.
				
			\subsubsection{Generalized equations}
				\label{sec:generalized_eqn}
				Applying KCL and KVL to one piece of figure \ref{fig:trans_line_lumped_model}, we get the following equations made up of partial differentials:
				\begin{align}
					\frac{\partial i(x,t)}{\partial x}&=G' v(x,t) + C'\frac{\partial v(x,t)}{\partial t}\\
					\frac{\partial v(x,t)}{\partial x}&= R' i(x,t)+L'\frac{\partial i(x,t)}{\partial t}
				\end{align}
				To remove the partial derivatives, we take another derivative of each equation and combine them:
				\begin{equation}
					\frac{\partial^2v}{\partial x^2} = R'G'v+(G'L'+R'C')\frac{\partial v}{\partial t}+C'L'\frac{\partial^2v}{\partial t^2}
					\label{eq:telegraph_equation}
				\end{equation}
				Equation \ref{eq:telegraph_equation} is called the \textbf{telegraph equation}. One thing to note is that it has more solutions than the two partial derivatives on which it is based. In practice this means that not every solution for equation \ref{eq:telegraph_equation} is a valid one. Solutions should always be checked against the base partial derivatives.
				
		\subsection{Current and voltage in sinusoidal steady state}
			\subsubsection{Phasor representation}
				In sinusoidal steady state, we the \textbf{phasor representation} for voltage and current in a transmission line. 
				\begin{align*}
					i(x,t)&=I_m(x)\cos(\omega t+\varphi_i(x))\\
					&=Re[I_m(x)e^{j\varphi_i(x)}e^{j\omega t}]\\
					&=Re[I(x)e^{j\omega t}]
				\end{align*}		
				Where $I(x)$ = complex amplitude or phasor representation of i(x,t)
				\begin{equation*}
					I(x)=I_m(x)e^{j\varphi_i(x)}
				\end{equation*}
				Where $I_m$ is the amplitude and $\varphi_i(x)$ is the phase of the current i. Both are real. The same can be done for voltage:
				\begin{align*}
					v(x,t)&=V_m(x)\cos(\omega t+\varphi_v(x))\\
					&=Re[V_m(x)e^{j\varphi_v(x)}e^{j\omega t}]\\
					&=Re[V(x)e^{j\omega t}]
				\end{align*}	
				
			\subsubsection{Why introduce phasors?}
				\label{sec:why_phasors}
				By introducing phasors into the base equations defined in section \ref{sec:generalized_eqn} we can simplify into:
				\begin{align}
					\frac{dI(x)}{dx}&=(G'+j\omega C')V(x)\\
					\frac{dV(x)}{dx}&=(R'+j\omega L')I(x)
				\end{align}
				Deriving and re-factoring these again gives us the following equations:
				\begin{align}
					\frac{d^2V}{dx^2}&=(G'+j\omega C')(R'+j\omega L')V =\gamma^2V\\
					\frac{d^2I}{dx^2}&=(G'+j\omega C')(R'+j\omega L')I = \gamma^2I
				\end{align}
				Here, we use $\gamma$, the \textbf{complex propagation constant}, as a way of simplifying notation. We can also rewrite it as:
				\begin{equation}
					\gamma = \alpha +j\beta
				\end{equation}
				Where $\alpha$ is the \textbf{attenuation constant} and $\beta$ is the \textbf{phase constant}.
			
			\subsubsection{Solution for sinusoidal steady state}
				By working solving the differential equations introduced in section \ref{sec:why_phasors}, and introducing the \textbf{characteristic impedance of the line}:
				\begin{equation}
					Z_0 = \sqrt{\frac{R+j\omega L'}{G'+j\omega C'}}
				\end{equation}
				we get a general solution:
				\begin{align}
					V&=\frac{V_L+Z_0I_L}{2}e^{\gamma x}+\frac{V_L-Z_0I_L}{2}e^{-\gamma x}\\
					I&=\frac{V_L+Z_0I_L}{2Z_0}e^{\gamma x}-\frac{V_L-Z_0I_L}{2Z_0}e^{-\gamma x}
				\end{align}
\end{document}
